\documentclass{article}

\usepackage[utf8]{inputenc}
\usepackage{amsmath}

\title{Measure Theory}
\date{}

\newtheorem{theorem}{Theorem}

\theoremstyle{definition}
\newtheorem{definition}{Definition}

\newcommand[\Reals]{\mathbb{R}}
\newcommand[\Borel]{\mathcal{B}}
\newcommand[\To]{\rightarrow}

\begin{document}

\maketitle

\section{probability spaces}

Basic definitions. Monotonicity.

\begin{theorem}
\((\Omega, \Filt, \mu) \)
\begin{enumerate}
\item If \( A \subset \bigcup_{i = 1}^\infty A_i \), then \(\mu(A) \leq \sum_{i =1}^\infty \mu(A_i) \)
\item If \( A_1 \subset A_2 \subset \cdots \) and \( A = \bigcup_{i = 1}^\infty A_i\) then \(\mu(A_i) \uparrow \mu(A) \) 
\item If \( A_1 \supset A_2 \supset \cdots \) and \( A = \bigcap_{i = 1}^\infty A_i \) then \(\mu(A_i) \downarrow \mu(A) \) 
\end{enumerate}

\begin{proof}

\begin{enumerate}
\item Take \(A'_i = A \cap A_i\), \(B_1 = A'_1\), \(B_n = A'_n - \bigcup_{i =1}^{n-1} A_i\). 
Then \(B_i\) are disjoint, \(\bigcup_{i =1}^\infty B_i = A\) and \(B_i \subset A_i\) which gives 
\[\mu(A)=\sum_{i =1}^\infty\mu(B_i) \leq \sum_{i = 1}^\infty \mu(A_i)\]
\item Write \(B_1 = A_1\), \(B_i = A_i - A_{i-1}\). Then the \(B_i\)s are disjoint, and \(\bigcup_{i = 1}^\infty B_i = A\), so 
\[\mu(A) = \sum_{i =1}^\infty \mu(B_i) = \lim_{n \rightarrow \infty} \mu(B_n) = \lim_{n \rightarrow \infty} \mu(A_n)\]
\item \(A_1 - A_n \uparrow A_1 - A\) so \(\mu(A_1 - A_n) \uparrow \mu(A_1 - A)\). But \(\mu(A_1 - A_n) = \mu(A_1) - \mu(A_n)\) and similar for the RHS, so \(\mu(A_n) \downarrow \mu(A)\). 
\end{enumerate}

Discrete probability spaces. \(\sigma\)-field generated by a collection of subsets. 

\begin{definition}
\(F\) is called a Stieltjes measure function if 
\begin{enumerate}
\item \(F\) is nondecreasing 
\item \(F\) is continuous from above: \(\lim_{y \downarrow x} F(y) = F(x)\)
\end{enumerate}
\end{definition}

\begin{theorem}
Given a Stieltjes function \(F\), there is a unique measure \(\mu\) on \((\Reals, \Borel\) such that 
\[ \mu((a,b]) = F(b) - F(a) \]
\end{theorem}

\begin{definition}
\(\mathcal S\) is a semialgebra if 
\begin{enumerate}
\item it is closed under intersections
\item If \(S\in \mathcal S\), then \(S^c\) is a finite union of elements in \(\mathcal S\)
\end{enumerate}
\end{definition}

\begin{definition} 
\(\mathcal A\) is an algebra if it is closed under finite unions and complements
\end{definition}





\end{document}
